%
% Copyright � 2012 Peeter Joot.  All Rights Reserved.
% Licenced as described in the file LICENSE under the root directory of this GIT repository.
%

\chapter{Potential due to cylindrical distribution}
\index{potential!cylindrical}
\label{chap:cylinderPotential}
%\blogpage{http://sites.google.com/site/peeterjoot2/math2012/cylinderPotential.pdf}
%\date{Feb 27, 2012}

\section{Motivation}

Consider a cylindrical distribution of mass (or charge) as in \cref{fig:cylinderPotential:cylinderPotentialFig1}, with points in the cylinder given  by \(\Br' = (r', \theta ', z')\) coordinates, and the point of measurement of the potential measured at \(\Br = (r, 0, 0)\).

\imageFigure{../figures/classicalmechanics/cylinderPotentialFig1}{coordinates for evaluation of cylindrical potential}{fig:cylinderPotential:cylinderPotentialFig1}{0.2}

Our potential, for a uniform distribution, will be proportional to

\begin{equation}\label{eqn:cylinderPotential:10}
\begin{aligned}
\phi(r)
&= \int \frac{ dV'}{\Abs{\Br - \Br'}} \\
&=
\int_0^R r' dr' \int_0^{2 \pi } d\theta' \int_{-L}^L \frac{dz' }{\sqrt{(z')^2+ \Abs{r - r' e^{i \theta }}^2}}.
\end{aligned}
\end{equation}

\section{Attempting to evaluate the integrals}

With

\begin{equation}\label{eqn:cylinderPotential:30}
\int_{-L}^L \frac{1}{\sqrt{z^2+u^2}} \, dz=\log \left(\frac{L+\sqrt{L^2+u^2}}{-L+\sqrt{L^2+u^2}}\right)
\end{equation}

This is found to be

\begin{equation}\label{eqn:cylinderPotential:50}
\phi(\Br) = \int_0^R r' dr' \int_0^{2 \pi } d\theta'
\log\left(\frac
{L+\sqrt{L^2+ \Abs{r - r' e^{ i \theta }}^2}}
{-L+\sqrt{L^2+ \Abs{r - r' e^{ i \theta }}^2}}
\right)
\end{equation}

It is clear that we can not evaluate this limit directly for \(L \rightarrow \infty\) since that gives us \(\infty/0\) in the logarithm term.  Presuming this can be evaluated, we must have to evaluate the complete set of integrals first, then take the limit.  Based on the paper \href{http://www.ifi.unicamp.br/~assis/J-Electrostatics-V63-p1115-1131(2005).pdf}{http://www.ifi.unicamp.br/~assis/J-Electrostatics-V63-p1115-1131(2005).pdf}, it appears that this can be evaluated, however, the approach used therein uses mathematics a great deal more sophisticated than I can grasp without a lot of study.

Can we proceed blindly using computational tools to do the work?  Attempting to evaluate the remaining integrals in \nbref{psIIp4InfCylPot.cdf} fails, since evaluation of both
 %\href{https://raw.github.com/peeterjoot/physicsplay/master/notes/phy354/problemSetIIproblem4IntegralsEvaluatingInfiniteCylindricalPotential.cdf}{Mathematica}

\begin{equation}\label{eqn:cylinderPotential:70}
\int_0^R r dr \log \left(a+\sqrt{a^2+r^2+b^2-2 r b \cos (\theta )}\right)
\end{equation}

and

\begin{equation}\label{eqn:cylinderPotential:90}
\int_0^{2 \pi }
d\theta
\log \left(a+\sqrt{a^2+r^2+b^2-2 r b \cos (\theta )}\right) ,
\end{equation}

either time out, or take long enough that I aborted the attempt to let them evaluate.

\subsection{Alternate evaluation order?}

We can also attempt to evaluate this by integrating in different orders.  We can for example do the \(r'\) coordinate integral first

\begin{equation}\label{eqn:cylinderPotential:110}
\begin{aligned}
\int_0^R
&dr
\frac{r}{\sqrt{a^2+r^2-2 r b \cos (\theta )+b^2}}  \\
&=
\sqrt{a^2+b^2-2 b R \cos (\theta )+R^2} -\sqrt{a^2+b^2} \\
&+
b \cos (\theta ) \log \left(
\frac{
\sqrt{a^2+b^2-2 b R \cos (\theta )+R^2}-b \cos (\theta )+R}
{
\sqrt{a^2+b^2}-b \cos (\theta )
}
\right).
\end{aligned}
\end{equation}

It should be noted that this returns a number of hard to comprehend ConditionalExpression terms, so care manipulating this expression may also be required.

If we try the angular integral first, we get

\begin{equation}\label{eqn:cylinderPotential:130}
\int_0^{2 \pi }
d\theta
\frac{1}{\sqrt{a^2+r^2-2 r b \cos (\theta )+b^2}}
=
\frac{2 K\left(-\frac{4 b r}{a^2+(b-r)^2}\right)}{\sqrt{a^2+(b-r)^2}}+\frac{2 K\left(\frac{4 b r}{a^2+(b+r)^2}\right)}{\sqrt{a^2+(b+r)^2}}
\end{equation}

where

\begin{equation}\label{eqn:cylinderPotential:150}
K(m) = F\left( \left.\frac{\pi }{2}\right| m \right)
\end{equation}

is the complete elliptic integral of the first kind.  Actually evaluating this integral, especially in the limiting case, probably requires stepping back and thinking a bit (or a lot) instead of blindly trying to evaluate.
