%
% Copyright � 2012 Peeter Joot.  All Rights Reserved.
% Licenced as described in the file LICENSE under the root directory of this GIT repository.
%

%\chapter{Classical Mechanics Euler Angles (lecture for phy354 taught by Prof E. Poppitz)}
\chapter{Classical Mechanics Euler Angles}
\index{Euler angles}
\label{chap:classicalMechanicsEulerAngles}
%\blogpage{http://sites.google.com/site/peeterjoot2/math2012/classicalMechanicsEulerAngles.pdf}
%\date{Mar 21, 2012}
%\gitRevisionInfo{classicalMechanicsEulerAngles}
%\keywords{Euler angles, general 3D rotation, angular velocity, phy354}

\section{Pictorially}

We want to look at some of the trig behind expressing general rotations.  We can perform a general rotation by a sequence of successive rotations.  One such sequence is a rotation around the \(z,x,z\) axes in sequence.  Application of a rotation of angle \(\phi\) takes us from our original \cref{fig:classicalMechanicsEulerAngles:classicalMechanicsEulerAnglesFig1} frame to that of \cref{fig:classicalMechanicsEulerAngles:classicalMechanicsEulerAnglesFig2}.  A second rotation around the (new) \(x\) axis by angle \(\theta\) takes us to \cref{fig:classicalMechanicsEulerAngles:classicalMechanicsEulerAnglesFig3}, and finally a rotation of \(\psi\) around the (new) \(z\) axis, takes us to \cref{fig:classicalMechanicsEulerAngles:classicalMechanicsEulerAnglesFig4}.

A composite image of all of these rotations taken together can be found in \cref{fig:classicalMechanicsEulerAngles:classicalMechanicsEulerAnglesFig5}.

\imageFigure{../figures/classicalmechanics/classicalMechanicsEulerAnglesFig1}{Initial frame}{fig:classicalMechanicsEulerAngles:classicalMechanicsEulerAnglesFig1}{0.2}
\imageFigure{../figures/classicalmechanics/classicalMechanicsEulerAnglesFig2}{Rotation by \(\phi\) around \(z\) axis}{fig:classicalMechanicsEulerAngles:classicalMechanicsEulerAnglesFig2}{0.2}
\imageFigure{../figures/classicalmechanics/classicalMechanicsEulerAnglesFig3}{Rotation of \(\theta\) around (new) \(x\) axis}{fig:classicalMechanicsEulerAngles:classicalMechanicsEulerAnglesFig3}{0.2}
\imageFigure{../figures/classicalmechanics/classicalMechanicsEulerAnglesFig4}{Rotation of \(\psi\) around (new) \(z\) axis}{fig:classicalMechanicsEulerAngles:classicalMechanicsEulerAnglesFig4}{0.2}

\imageFigure{../figures/classicalmechanics/classicalMechanicsEulerAnglesFig5}{All three rotations superimposed}{fig:classicalMechanicsEulerAngles:classicalMechanicsEulerAnglesFig5}{0.2}

\section{Relating the two pairs of coordinate systems}

Let us look at this algebraically instead, using \cref{fig:classicalMechanicsEulerAngles:classicalMechanicsEulerAnglesFig6} as a guide.

\imageFigure{../figures/classicalmechanics/classicalMechanicsEulerAnglesFig6}{A point in two coordinate systems}{fig:classicalMechanicsEulerAngles:classicalMechanicsEulerAnglesFig6}{0.2}

Step 1.  Rotation of \(\phi\) around \(z\)

\begin{equation}\label{eqn:classicalMechanicsEulerAngles:20}
\begin{bmatrix}
x' \\
y' \\
z'
\end{bmatrix}
=
\begin{bmatrix}
\cos\phi & \sin\phi & 0 \\
-\sin\phi & \cos\phi & 0 \\
0 & 0 & 1
\end{bmatrix}
\begin{bmatrix}
x \\
y \\
z
\end{bmatrix}
\end{equation}

Step 2.  Rotation around \(x'\).

\begin{equation}\label{eqn:classicalMechanicsEulerAngles:40}
\begin{bmatrix}
x'' \\
y'' \\
z''
\end{bmatrix}
=
\begin{bmatrix}
1 & 0 & 0 \\
0 & \cos\theta & \sin\theta & 0 \\
0 & -\sin\theta & \cos\theta & 0
\end{bmatrix}
\begin{bmatrix}
x' \\
y' \\
z'
\end{bmatrix}
\end{equation}

Step 3.  Rotation around \(z''\).

\begin{equation}\label{eqn:classicalMechanicsEulerAngles:60}
\begin{bmatrix}
x''' \\
y''' \\
z'''
\end{bmatrix}
=
\begin{bmatrix}
\cos\psi & \sin\psi & 0 \\
-\sin\psi & \cos\psi & 0 \\
0 & 0 & 1
\end{bmatrix}
\begin{bmatrix}
x'' \\
y'' \\
z''
\end{bmatrix}
\end{equation}

So, our full rotation is the composition of the rotation matrices

\begin{equation}\label{eqn:classicalMechanicsEulerAngles:80}
\begin{bmatrix}
x''' \\
y''' \\
z'''
\end{bmatrix}
=
\begin{bmatrix}
\cos\psi & \sin\psi & 0 \\
-\sin\psi & \cos\psi & 0 \\
0 & 0 & 1
\end{bmatrix}
\begin{bmatrix}
1 & 0 & 0 \\
0 & \cos\theta & \sin\theta & 0 \\
0 & -\sin\theta & \cos\theta & 0
\end{bmatrix}
\begin{bmatrix}
\cos\phi & \sin\phi & 0 \\
-\sin\phi & \cos\phi & 0 \\
0 & 0 & 1
\end{bmatrix}
\begin{bmatrix}
x \\
y \\
z
\end{bmatrix}
\end{equation}

Let us introduce some notation and write this as

\begin{equation}\label{eqn:classicalMechanicsEulerAngles:100}
B_z(\alpha)
=
\begin{bmatrix}
\cos\alpha & \sin\alpha & 0 \\
-\sin\alpha & \cos\alpha & 0 \\
0 & 0 & 1
\end{bmatrix}
\end{equation}

\begin{equation}\label{eqn:classicalMechanicsEulerAngles:120}
B_x(\theta) =
\begin{bmatrix}
1 & 0 & 0 \\
0 & \cos\theta & \sin\theta & 0 \\
0 & -\sin\theta & \cos\theta & 0
\end{bmatrix}
\end{equation}

so that we have the mapping

\begin{equation}\label{eqn:classicalMechanicsEulerAngles:140}
\Br \rightarrow B_z(\psi) B_x(\theta) B_z(\phi) \Br
\end{equation}

Now let us write

\begin{equation}\label{eqn:classicalMechanicsEulerAngles:160}
\Br =
\begin{bmatrix}
x_1 \\
x_2 \\
x_3
\end{bmatrix}
\end{equation}

We will call

\begin{equation}\label{eqn:classicalMechanicsEulerAngles:180}
A(\psi, \theta, \phi) = B_z(\psi) B_x(\theta) B_z(\phi)
\end{equation}

so that

\begin{equation}\label{eqn:classicalMechanicsEulerAngles:200}
x_i' = \sum_{j = 1}^3 A_{ij} x_j.
\end{equation}

We will drop the explicit summation sign, so that the summation over repeated indices are implied

\begin{equation}\label{eqn:classicalMechanicsEulerAngles:220}
x_i' = A_{ij} x_j.
\end{equation}

This matrix \(A(\psi, \theta, \phi)\) is in fact a general parameterization of the \(3 \times 3\) special orthogonal matrices. The set of three angles \(\theta\), \(\phi\), \(\psi\) parameterizes all rotations in 3dd space.  Transformations that preserve \(\Ba \cdot \Bb\) and have unit determinant.

In symbols we must have

\begin{equation}\label{eqn:classicalMechanicsEulerAngles:240}
A^\T A = 1
\end{equation}
\begin{equation}\label{eqn:classicalMechanicsEulerAngles:260}
\det A = +1.
\end{equation}

Having solved this auxiliary problem, we now want to compute the angular velocity.

We want to know how to express the coordinates of a point that is fixed in the body.  i.e. We are fixing \(x_i'\) and now looking for \(x_i\).

The coordinates of a point that has \(x'\), \(y'\) and \(z'\) in a body-fixed frame, in the fixed frame are \(x\), \(y\), \(z\).  That is given by just inverting the matrix

\begin{equation}\label{eqn:classicalMechanicsEulerAngles:480}
\begin{aligned}
\begin{bmatrix}
x_1 \\
x_2 \\
x_3
\end{bmatrix}
&=
A^{-1}(\psi, \theta, \phi)
\begin{bmatrix}
x_1' \\
x_2' \\
x_3'
\end{bmatrix} \\
&=
B_z^{-1}(\phi)
B_x^{-1}(\theta)
B_z^{-1}(\psi)
\begin{bmatrix}
x_1' \\
x_2' \\
x_3'
\end{bmatrix} \\
&=
B_z^{\T}(\phi)
B_x^{\T}(\theta)
B_z^{\T}(\psi)
\begin{bmatrix}
x_1' \\
x_2' \\
x_3'
\end{bmatrix}
\end{aligned}
\end{equation}

Here we have used the fact that \(B_x\) and \(B_z\) are orthogonal, so that their inverses are just their transposes.

We have finally

\begin{equation}\label{eqn:classicalMechanicsEulerAngles:280}
\begin{bmatrix}
x_1 \\
x_2 \\
x_3
\end{bmatrix}
=
B_z(-\phi)
B_x(-\theta)
B_z(-\psi)
\begin{bmatrix}
x_1' \\
x_2' \\
x_3'
\end{bmatrix}
\end{equation}

If we assume that \(\psi\), \(\theta\) and \(\phi\) are functions of time, and compute \(d \Br/dt\).  Starting with

\begin{equation}\label{eqn:classicalMechanicsEulerAngles:300}
x_i = [A^{-1}(\phi, \theta, \psi)]_{ij} x_j',
\end{equation}


\begin{equation}\label{eqn:classicalMechanicsEulerAngles:320}
\Delta x_i =
\left(
[A^{-1}(\phi + \Delta \phi, \theta + \Delta \theta, \psi + \Delta \psi)]_{ij}
-[A^{-1}(\phi, \theta, \psi)]_{ij}
\right)
x_j'.
\end{equation}

For small changes, we can Taylor expand and retain only the first order terms.  Doing that and dividing by \(\Delta t\) we have

\begin{equation}\label{eqn:classicalMechanicsEulerAngles:340}
\frac{d x_i}{dt} =
\left(
\PD{\psi}{} A^{-1}_{ij} \dot{\psi}
+\PD{\theta}{} A^{-1}_{ij} \dot{\theta}
+\PD{\phi}{} A^{-1}_{ij} \dot{\phi}
\right) x_j'
\end{equation}

Now, we use

\begin{equation}\label{eqn:classicalMechanicsEulerAngles:360}
x_j' = A_{jl} x_l,
\end{equation}

so that we have

\begin{equation}\label{eqn:classicalMechanicsEulerAngles:380}
\frac{d x_i}{dt} =
\left(
\left(\PD{\psi}{} A^{-1}_{ij} \right) A_{jl} \dot{\psi}
+\left(\PD{\theta}{} A^{-1}_{ij} \right) A_{jl} \dot{\theta}
+\left(\PD{\phi}{} A^{-1}_{ij} \right) A_{jl} \dot{\phi}
\right) x_l.
\end{equation}

We are looking for a relation of the form

\begin{equation}\label{eqn:classicalMechanicsEulerAngles:400}
\frac{d\Br}{dt} = \BOmega \cross \Br.
\end{equation}

We can write this as

\begin{equation}\label{eqn:classicalMechanicsEulerAngles:420}
\begin{bmatrix}
v_x \\
v_y \\
v_z
\end{bmatrix}
=
\left(
\thetadot \PD{\theta}{A^{-1}} A
+\phidot \PD{\phi}{A^{-1}} A
+\psidot \PD{\psi}{A^{-1}} A
\right)
\begin{bmatrix}
x \\
y \\
z
\end{bmatrix}
\end{equation}

Actually doing this calculation is asked of us in HW6.  The final answer is

\begin{equation}\label{eqn:classicalMechanicsEulerAngles:440}
\frac{d x_i}{dt} = \left(
\phidot \epsilon_{ijk} j_n^\phi
+\thetadot \epsilon_{ijk} j_n^\theta
+\psidot \epsilon_{ijk} j_n^\psi
\right) x_j.
\end{equation}

Here \(\epsilon_{ijk}\) is the usual fully antisymmetric tensor with properties

\begin{equation}\label{eqn:classicalMechanicsEulerAngles:460}
\epsilon_{ijk} =
\left\{
\begin{array}{l l}
0 & \quad \mbox{when any of the indices are equal.} \\
1 & \quad \mbox{for any of \(ijk = 123, 231, 312\) (cyclic permutations of \(123\).} \\
-1 & \quad \mbox{for any of \(ijk = 213, 132, 321\).} \\
\end{array}
\right.
\end{equation}
